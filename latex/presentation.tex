\documentclass{beamer}
\usetheme{Boadilla}

\title{UCLA Extension Machine Learning:\\
	Recipe Naming Using Text Analysis}

\begin{document}

\begin{frame}
\titlepage
\end{frame}


\begin{frame}
	\frametitle{Concept and Data Source}

	Concept: Given a recipe (consisting of a list of ingredients and a set
	of corresponding instruction set), produce an accurate title for the
	recipe.\\
	\medskip
	Main data source:
	\begin{itemize}
		\item ~250,000 recipes scraped from various websites
			(foodnetwork.com, epicurious.com, allrecipes.com). The
			program that created this set is MIT lisenced and was
			created by Ryan T. Lee (github: rtlee9), and the
			dataset used is ODC lisenced.
			
		\item Dataset is a json file with recipe key and corresponding
			title, ingredient list, and instruction set.
	\end{itemize}
\end{frame}
			

\begin{frame}
	\frametitle{Basic concept: vectorizing text}
	\begin{itemize}
		\item The typical starting point for text-based machine
			learning is to look at the words in a text, and how
			those words correspond to features that you are trying
			to predict.
		\item In order to do any form of efficient computational
			analysis, the text that you want to analyze must be
			turned into a vector. 
		\item There exist many different ways to vectorize text.
			Methods to vectorize text based on letters, words,
			relationships between words, proximity between words,
			etc.

	\end{itemize}
\end{frame}

\begin{frame}

	\frametitle{What does a text vector look like?}
	\begin{enumerate}
	
		\item There exist many different ways to vectorize text.
			Methods to vectorize text based on letters, words,
			relationships between words, proximity between words,
			etc.

	\end{enumerate}
\end{frame}


\end{document}
