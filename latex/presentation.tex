\documentclass{beamer}
\usetheme{Boadilla}

\title{Recipe Naming Using Text Analysis}
\author{Elliot Lehman}
\institute{UCLA Extension}
\begin{document}

\begin{frame}
\titlepage
\end{frame}


\begin{frame}
	\frametitle{Concept and data source}

	Concept: Given a recipe (consisting of a list of ingredients and a set
	of corresponding instruction set), produce an accurate title for the
	recipe.\\
	\medskip
	Main data source:
	\begin{itemize}
		\item ~250,000 recipes scraped from various websites
			(foodnetwork.com, epicurious.com, allrecipes.com). The
			program that created this set is MIT lisenced and was
			created by Ryan T. Lee (github: rtlee9), and the
			dataset used is ODC lisenced.
			
		\item Dataset is a json file with recipe key and corresponding
			title, ingredient list, and instruction set.
	\end{itemize}
\end{frame}
			

\begin{frame}
	\frametitle{Basic concept: vectorizing text}
	\begin{itemize}
		\item The typical starting point for text-based machine
			learning is to look at the words in a text, and how
			those words correspond to features that you are trying
			to predict.
		\item In order to do any form of efficient computational
			analysis, the text that you want to analyze must be
			turned into a vector. 
		\item There exist many different ways to vectorize text.
			Methods to vectorize text based on letters, words,
			relationships between words, proximity between words,
			etc.

	\end{itemize}
	\bigskip


	\noindent\fbox{%
    	\parbox{\textwidth}{%
	\begin{itemize}
		\item The goal is to input a \textit{recipe vector} and output
			a \textit{title vector}. 

			\medskip
			Note: We will treat these as
			different vector spaces so, in general, a \textit{recipe
			vector} does not equal a \textit{title vector}.
	\end{itemize}
    		}%
	}
\end{frame}

\begin{frame}
	\frametitle{A toy example of text vectorization}
	Given the below dataset, what would our \textit{recipe vector} and \textit{title vector} look like?
	\begin{center}
		\begin{tabular}{ |c c | c|}
			\hline
			Recipe & & Title \\
			\hline\hline
			bread & avocado & avocado toast \\
			\hline
			lettuce & avocado & salad \\		
			\hline
		\end{tabular}
	\end{center}

	\begin{enumerate}
	
		\item Define your basis vectors.
		\begin{itemize}
			\item There are 3 unique words in the Recipe section,
				meaning that any recipe using ingredients in
				this data set can be represented by a vector of
				size 3.
			\item The word ``avocado'' would be represented by the vector $(1,0,0)_r$, 
				``bread'' by $(0,1,0)_r$, and lettuce by $(0,0,1)_r$
		\end{itemize}
		\item Vectors of recipes.
		\begin{itemize}
			\item Now that basis vectors are defined, we can add them to get recipes.
			\item The phrase ``avocado lettuce'' is therefore represented by $(1,0,1)_r$.
			\itme The phrase ``avocado bread lettuce'' is the vector $(1,1,1)_r$.
		\end{itemize}

	\end{enumerate}
\end{frame}

\begin{frame}
	\frametitle{We have vectors... now what?}

\end{frame}

\end{document}
